%%%%%%%%%%%%%%%%%%%%%%%%%%%%%%%%%%%%%%%%%
% University Assignment Title Page 
% LaTeX Template
%
% This template has been downloaded from:
% http://www.latextemplates.com
%
% Original author:
% WikiBooks (http://en.wikibooks.org/wiki/LaTeX/Title_Creation)
% 
% Instructions for using this template:
% This title page is presently capable of being compiled as is. This is not 
% useful for including it in another document. To do this, you have two options: 
%
% 1) Copy/paste everything between \begin{document} and \end{document} 
% starting at \begin{titlepage} and paste this into another LaTeX file where you 
% want your title page.
% OR
% 2) Remove everything outside the \begin{titlepage} and \end{titlepage} and 
% move this file to the same directory as the LaTeX file you wish to add it to. 
% Then add \input{./title_page_1.tex} to your LaTeX file where you want your
% title page.
%
%%%%%%%%%%%%%%%%%%%%%%%%%%%%%%%%%%%%%%%%%
\documentclass[12pt]{article}
\usepackage{geometry}                		% See geometry.pdf to learn the layout options. There are lots.
\geometry{letterpaper}                   		% ... or a4paper or a5paper or ... 
%\geometry{landscape}                		% Activate for for rotated page geometry
%\usepackage[parfill]{parskip}    		% Activate to begin paragraphs with an empty line rather than an indent
\usepackage{graphicx}				% Use pdf, png, jpg, or eps� with pdflatex; use eps in DVI mode
								% TeX will automatically convert eps --> pdf in pdflatex		
\usepackage{amssymb}
\setlength{\parindent}{0pt}

%----------------------------------------------------------------------------------------
%	PACKAGES AND OTHER DOCUMENT CONFIGURATIONS
%----------------------------------------------------------------------------------------


\begin{document}

\begin{titlepage}

\newcommand{\HRule}{\rule{\linewidth}{0.5mm}} % Defines a new command for the horizontal lines, change thickness here

\center % Center everything on the page
 
%----------------------------------------------------------------------------------------
%	HEADING SECTIONS
%----------------------------------------------------------------------------------------

\textsc{\LARGE}\\[1.5cm] % Name of your university/college
\textsc{\Large Computer System Engineering}\\[0.5cm] % Major heading such as course name
\textsc{\large Design Project 2 Executive Summary}\\[0.5cm] % Minor heading such as course title

%----------------------------------------------------------------------------------------
%	TITLE SECTION
%----------------------------------------------------------------------------------------

\HRule \\[0.4cm]
{ \huge \bfseries Ad-Hoc Wireless Network}\\[0.4cm] % Title of your document
\HRule \\[1.5cm]
 
%----------------------------------------------------------------------------------------
%	AUTHOR SECTION
%----------------------------------------------------------------------------------------

% If you don't want a supervisor, uncomment the two lines below and remove the section above
\Large \emph{Author:}\\
Bo Song  \textsc{11302010003}\\% Your name
TianHao Wang  \textsc{11302010005}\\% Your name
XiaoBin Xu  \textsc{11302010008}\\[3cm]% Your name

%----------------------------------------------------------------------------------------
%	DATE SECTION
%----------------------------------------------------------------------------------------

{\large \today}\\[3cm] % Date, change the \today to a set date if you want to be precise

%----------------------------------------------------------------------------------------
%	LOGO SECTION
%----------------------------------------------------------------------------------------

%\includegraphics{Logo}\\[1cm] % Include a department/university logo - this will require the graphicx package
 
%----------------------------------------------------------------------------------------

\vfill % Fill the rest of the page with whitespace

\end{titlepage}

\section*{1~~~~Introduction}
\section*{2~~~~Design Description}

The design is partitioned into two parts, since the link layer is simple, we give the algorithm part in the network layer, and the protocol part in the end-to-end layer.

\section*{2.1~~~~Routing Algorithm}

In this section, we give our routing algorithm in the network layer. We first describe the data structure, that is, what we add in the header and tailer of the packet. Then we describe how the algorithm works.

\section*{2.1.0~~~~Some Terms}
We define the base station as node D, the handheld device which is about to send information as the node S. As mentioned in DP description, every handheld device has the location information of base station. Then we define a predefined constant P.\\[1em]

\section*{2.1.1~~~~Data Structure}
message meta data:\\
origin source time
ok node list
path node list
id
\\
message content: image(optional) and origin source location information

\section*{2.1.2~~~~Adaptive Effective Zone Algorithm}
In order to let messages reach the base station and maximize the throughput, we adopt a location based algorithm instead of brute force algorithm to broadcast messages.\\[1em]

The description mentioned below only focus on the whole process of routing and ignore the detail of other things such as reliability, security issues etc.\\[1em]

\begin{enumerate}
\item S invoke the predefined scan() function to get a list of tuples (node, loss\_prob), for the sake of improving effectivity, then S chooses those nodes whose loss\_prob is lower than P, and who is not in the origin path.\\[1em]
\item S broadcasts a message which has an unique ID to nearby nodes. The message contains the location information of S, a list which contains nodes chosen in Step1, the content information which contains location information of origin source node(here is S) or image information.\\[1em]
\item If node A receives the message sent by S, the first thing is to check whether the message has been sent from A, and discard it if it has. Then, check whether S wants A to broadcast the message by checking the node list. If not, discard it, too. \\[1em]
\item Then A calculates whether A is in the effective zone calculated using the location information of the sender and the base station, if A is not in the effective zone, discard it. Otherwise it rebroadcast the content message along with updated mete data as S does and invokes receive().\\[1em]
\item If all things go well, after the  iterative processes of Step 1 to 5, the message will reach D with relative small number of hops.
\end{enumerate}

\section*{2.2~~~~Transfer Protocol}

\section*{2.3~~~~Different Reliability}
\section*{2.3.1~~~~Basic Lost Handling}
In our location based algorithm, many nodes discard the packet for the sake of improving throughput. As a result, there will be a problem that S can not reach D through the nodes in the effective zone. Therefore, we need a basic mechanism to handle such problem along with the lost problem in send or receive process.\\[1em]

If the sender S does not receive confirm information carried by receive() function for a while, there are two possibilities:1. 2.
(rebroadcast cast for first several times and then expand effective zone gradually)
\section*{2.3.2~~~~Burst Mode}
 location information 5min, accelerate the frequency of rebroadcast
\section*{2.3.3~~~~Image Reliability}
try to send most recent image to base station

\section*{2.4~~~~Malicious Nodes Handling}
Nodes in the system should distinguish malicious nodes by some strange behavior, so we classify different kinds of malicious nodes and handle it separately.
\subsection*{Type One}
the malicious node hear a legitimate message, then broadcasts many copy(such as one million) of this message pretending it received it from the source.
\paragraph{How to handle}
We add a counter to every node. When a node receive abnormal number of messages of one sender, the node can mark sender as a malicious node and discard message from it. \\In this situation, we effectively prevent the millions copies of message from propagating forward to the whole system. However, nodes around the malicious node will suffer millions of connections till the malicious node stop sending, so these nodes will lose the ability to receive other messages. The only thing we can do to handle this relative small problem is send warning message to the base station or the other people to handle the malicious node in physical way. 
\subsection*{Type TWO}
The malicious node A receive a message, send the receive() function to the sender but stop forward the message and the sender will consider A will do its job as usual. 
\paragraph{How to handle}
Under most situation, it is not a big problem since other nodes around sender will do their job as usual.\\ However, when all valid nodes(in effective zone and low lost probability) around sender are malicious node, the sender can not send message to the base station. We will disscuse it further in our report.
\subsection*{Type THREE}
The malicious node change the message and send forward to the base station. 
\paragraph{How to handle}
The method is easy that encrypt the message, but how to implement it correctly and prevent malicious node decrypt is a difficult theoretical problem and we will not discuss here.
\section*{3~~~~Conclusion}

Word Count:2483(exclude title page)

\end{document}